\section{实验题目}

\textbf{利用 MMX \footnotemark 加速图片的渐入渐出效果。}

\footnotetext{MMX: 由英特尔开发的一种SIMD多媒体指令集,共有57条指令。它于1996年集成在英特尔奔腾(Pentium)MMX处理器上,以提高其多媒体数据的处理能力。\cite{furht1997multimedia}}

渐入渐出 (Fade In \& Fade Out) 是一种常见的图像处理特效。

其原理用一个加权平均公式即可表示:

\begin{equation} \label{eq:1}
p = p_1 w + p_2 (1 - w) = w(p_1 - p_2) + p_2 ,( w \in [0, 1] )
\end{equation}

其中,$p, p_1, p_2$ 都是颜色,称$p_1, p_2$为源色,$p$为合成色。

这个公式表示如何\textbf{按照比例 $w$ 融合两种颜色}。

渐入渐出过程是从0\%(100\%)开始迭代递增(减) $w$ 时伴随着的 $p$ 的变化过程。在这个过程中,一种源色在合成色中的比例不断增加直至100\%,另一种则不断减少直至0\%。

\section{实验设计}

在计算机中,图像以像素阵列的形式存储,只要能按像素读写图片,就能实现颜色融合。进而配合延时控制,即可实现渐入渐出效果。

不使用 MMX 也可以实现渐入渐出,但 MMX 可以利用\textbf{并发}加速\textbf{颜色融合}这个过程。

以 24 位位图作为研究对象,因为其色元数据具有 8 位对齐的特征 \footnotemark ,其每个像素用 24 bits (3 Bytes) 表示,默认格式为 RGB。红(Red)、绿(Green)、蓝(Blue) 依次用像素的一个 Byte 来表示,取值范围为 0 - 255。

\footnotetext{MMX 只能正确处理已经对齐的数据,否则需要耗费大量的时间做数据对齐。}

按照算法,$p, p_1, p_2$都是一个颜色向量。三种原色的融合要分开做,不能一起做。一个简单的反例是,白色(0xFFFFFF)与黑色(0x000000)以50\%的比例融合的时候,如果将颜色直接代入融合公式:

(0xFFFFFF - 0x000000) / 2 + 0x000000 = 0x7FFFFF

得到的是青色(0x7FFFFF),这与预期的灰色(0x7F7F7F)不符。

正确的做法是将三原色分别代入融合公式:


(0xFF - 0x00) / 2 + 0x00 = 0x7F


这样才能得到灰色(0x7F7F7F)。

所以颜色融合本质上是\textbf{向量操作},涉及向量的加减以及数乘运算。

MMX 指令集正好就提供了并发进行向量操作的指令, excited!
