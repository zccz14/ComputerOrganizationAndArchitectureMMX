\section{算法设计}

在 24bpp \footnotemark 位图中,一个色元恰好占据一个字节,而寄存器的长度数倍于此,一个色元单独使用一个寄存器,于时间、空间都有些浪费,希望能用一个寄存器,一个指令,同时处理多个色元。而且数据正好是按字节对齐的,非常符合 MMX 的优化条件。

\footnotetext{24 bits per pixel: 每像素 24 位,即 24 位位图。}

由公式 \ref{eq:1}得色元融合需要对色元做一个乘法,因此处理色元乘法需要一个字长的空间,因此 64 位寄存器最多可能一次性处理 4 个色元。

可以一次取出 4 个色元放置在64位寄存器的低32位,利用之前 \ref{pack} 提过的解压缩指令,将其分散到 64 位寄存器上的 4 个字中,然后按照公式 \ref{eq:1} 给出的算法进行运算。

设 $\max{F} = 32767$,对于渐变值 $F \in [0, \max{F}]$,给定像素色元 $A, B \in [0, 255]$

\begin{equation} \label{eq:2}
\begin{split}
    \frac{F A + (\max{F} - F) B}{\max{F}}
    & = 
    \frac{2 F A + 2 (\max{F} - F)B}{2\max{F}} \\
    & \approx 
    (\frac{F A}{2 (\max{F} + 1)} + \frac{(\max{F} - F) B}{2 (\max{F} + 1)}) \times 2 \\
    & = 2(\text{mulhw}(A, F) + \text{mulhw}(B, \max{F} - F))
\end{split}
\end{equation}

使渐变系数为一个有符号非负整型字的,最大值为 0x7FFF,渐变分为 32768 个级别。

在乘法中,两个字相乘得到一个双字长数据,pmulhw是有符号乘法取结果双字中的高位字,相当于相乘以后除以 65536,因此在算法公式 \ref{eq:2} 中,要刻意凑出分母 65536,这样才可以替换出 pmulhw。

\section{代码实现}

\lstinputlisting[language={[visual]C++}, caption={MMX 图像融合}]{code/ImageFusionMMX.cpp}

\section{性能基准测试}

\lstinputlisting[language={[visual]C++}, caption={基准测试}]{code/Benchmark.cpp}

利用 Visual Studio 的性能探查器可以检测 CPU 使用率。

运行测试约 1 分钟后终止检测,得到函数运行独占样本数:

\begin{center}
\begin{tabular}{|c|c|}
    \hline
    函数名 & 独占样本数 \\
    \hline
    ImageFusion & 45049 \\
    \hline
    ImageFusionMMX & 15633 \\
    \hline
\end{tabular}
\end{center}

得到近似加速比:

$$
S = \frac{45049}{15633} = 2.88 \footnotemark
$$

\footnotetext{实际数据根据不同的实验环境会有所变化,正常范围是 2.0 - 4.0。}
