\section{概述}

\subsection{关键字}

\begin{itemize}
    \item Byte: 字节 (8 bits)
    \item Word: 字 (16 bits)
    \item Doubleword: 双字 (32 bits)
    \item Quadword: 四字 (64 bits)
    \item Double Quadword: 八字 (128 bits)
    \item Pack: 压缩
    \item Unpack: 解压
    \item Saturation: 饱和
\end{itemize}

\subsection{饱和}

饱和(Saturation)是 MMX 的一个很有特点的运算性质,它能在运算结果溢出时直接取上下界的值作为运算结果。

在无符号饱和字运算中,上溢时取结果为 0xFFFF,下溢时取结果为 0x0000。

在带符号饱和字运算中,上溢时取结果为 0x7FFF (32767D),下溢时取结果为 0x8000 (-32768D)。

以 0x4000 * 0x4000 = 0x10000000 为例,一般的字运算直接截断高位字取低位字得到 0x0000 作为结果,而无符号饱和运算得到 0xFFFF,有符号饱和运算得到 0x7FFF。

\subsection{压缩} \label{pack} 

压缩(打包)/解压缩(解包) 在 MMX 中是一对很重要的操作,它们与 MMX 的效率密不可分。

以 PACKUSWB \footnote{PACKUSWB: Pack with Unsigned Saturation from Word to Byte} (无符号饱和压缩字到字节) 与 PUNPCKLBW \footnote{PUNPCKLBW: Packed Unpack Low-order Byte to Word} 指令为例说明在 MMX 中的压缩/解压缩操作。

指令的格式如下:

\begin{lstlisting}
PACKUSWB mm mm/m64
PUNPCKLBW mm mm/m32
\end{lstlisting}

mm 是 MMX 专用的 64 位寄存器格式,有 mm0, mm1, ..., mm7,共 8 个 64 位寄存器。m64 则表示 64 位内存空间格式,m32 则表示 32 位的内存空间格式。

第一个操作数被称为“目的操作数”,只能通过寄存器寻址;第二个操作数被称为“源操作数”,可以通过寄存器寻址也可以通过内存寻址。

压缩是采取 “有损压缩” 的策略。

PACKUSWB 的机理是先将 64 位操作数视作 4 个字的序列,分别进行无符号饱和压缩,得到 4 个字节;那么 2 个 64 位操作数就可以得到 8 个字节,即一个 64 位数据,然后赋值到目的操作数寄存器中,其中由源操作数得到的 4 个字节在结果中占据高 32 位,而目的操作数压缩得到的 4 个字节在结果中占据低 32 位。

解压缩采取 “交叉组合” (Interleave) 的策略。

PUNPCKLBW 的机理是先将 64 位操作数的低 32 位取出,看作 4 个字节;那么 2 个 64 位操作数可以得到 2 组 4个字节的数据;交叉放置这些字节得到 4 个字,即一个 64 位数据,然后赋值到目的操作数寄存器中,其中由源操作数得到的字节占据字的高位,由目的操作数得到的字节占据字的低位。

举一个具体的例子:

\begin{lstlisting}[language={[x86masm]Assembler}]
; mm0: 7FFF8000FFFF002B, mm1: 002D1121DEADBEEF
packuswb mm0, mm1
; 002D|1121|DEAD|BEEF|7FFF|8000|FFFF|002B
;   2D|  FF|  00|  00|  FF|  00|  00|  2B
; mm0: 2DFF0000FF00002B
punpcklbw mm0, mm1
; interleave mm1, mm0
; mm1's lower doubleword: DE|AD|BE|EF
; mm0's lower doubleword: FF|00|00|2B
; DE|FF|AD|00|BE|00|EF|2B
; mm0: DEFFAD00BE00EF2B
\end{lstlisting}


\section{指令说明}

\subsection{指令清单}

\subsubsection{拷贝指令}

\begin{itemize}
    \item movd: Move Doubleword 拷贝双字(32 bits)
    \item movq: Move Quadword 拷贝四字(64 bits)
\end{itemize}

\subsubsection{压缩指令}

解压缩 (Packed Unpack)


\begin{itemize}
    \item punpcklbw: Packed Unpack Low-order Bytes to Word 
    \item punpcklwd: Packed Unpack Low-order Words to Doubleword
    \item punpckldq: Packed Unpack Low-order Doublewords to Quadword
    \item punpcklqdq: Packed Unpack Low-order Quadwords to Double Quadword
    \item punpckhbw: Packed Unpack High-order Bytes to Word 
    \item punpckhwd: Packed Unpack High-order Words to Doubleword
    \item punpckhdq: Packed Unpack High-order Doublewords to Quadword
    \item punpckhqdq: Packed Unpack High-order Quadwords to Double Quadword
\end{itemize}

饱和压缩 (Pack with Saturation)

饱和策略分为\textbf{带符号饱和}与\textbf{无符号饱和}。

\begin{itemize}
    \item packuswb: Pack with Unsigned Saturation Word to Byte
    \item packusdw: Pack with Unsigned Saturation Doubleword to Word
    \item packsswb: Pack with Signed Saturation Word to Byte
    \item packssdw: Pack with Signed Saturation Doubleword to Word
\end{itemize}

\subsubsection{运算指令}

MMX的运算主要是指 “压缩运算” 。

\textbf{一般来说},指令名以 p 开头(packed),然后接着操作名(add, sub, mul等),然后是运算属性(u: Unsigned, s: Staturation, h: High-order, l: Low-order 等),最后是操作数的长度(b: Byte, w: Word, d: Doubleword, q: Quadword, dq: Double Quadword)。

以 add 为例:

\begin{itemize}
    \item paddb: Packed Add Bytes
    \item paddw: Packed Add Words
    \item paddd: Packed Add Doublewords
    \item paddq: Packed Add Quadwords
    \item paddsb: Packed Add with Signed Saturation Bytes
    \item paddsw: Packed Add with Signed Saturation Words
    \item paddusb: Packed Add with Unsigned Saturation Bytes
    \item paddusw: Packed Add with Unsigned Saturation Words
\end{itemize}

对于 mul 来说:

\begin{itemize}
    \item pmuldq: Packed Multiply Doublewords to Quadword
    \item pmulhrsw: Packed Multiply High with Round and Scale Words *
    \item pmulhuw: Packed Multiply High-order result of Unsigned Words
    \item pmulhw: Packed Multiply High-order result of signed Words
    \item pmulld: Packed Multiply Low-order result of signed Doublewords
    \item pmullw: Packed Multiply Low-order result of signed Words
    \item pmulludq: Packed Multiply Low-order result of Unsigned Doublewords to Quadwords
\end{itemize}